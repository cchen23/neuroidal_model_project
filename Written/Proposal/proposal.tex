\documentclass[pageno]{jpaper}

\newcommand{\IWreport}{2018}
\newcommand{\quotes}[1]{``#1''}


\widowpenalty=9999

\usepackage[normalem]{ulem}
\usepackage{scrextend}	% for the indented environment
\usepackage{amsmath}
\usepackage{graphicx, caption, subcaption}

\newenvironment{indented}{\begin{addmargin*}{1.5cm}}{\end{addmargin*}}

\begin{document}

\title{COS511 Project Proposal: Studying and Implementing Cortical Learning Algorithms}

\author{Cathy Chen and Stefan Keselj}
\date{\today}
\maketitle

Our project will focus on the neuroidal model, a biologically plausible computational model proposed by Leslie Valiant in 1994 \cite{valiant_circuits_1994}.

We will begin by studying the model itself and by comparing the mechanisms of the neuroidal model with hypotheses from the field of neuroscience about mechanisms of the human brain.

We will then turn to algorithms implemented on this model, particularly those proposed by Valiant in 2005 \cite{valiant_memorization_2005} and Papadimitriou in 2015 \cite{papadimitriou_cortical_2015}. In this phase of the project we seek to understand the theoretical performance of these algorithms (in terms of both computational complexity and learning ability). [TODO: WRITE ABOUT OVERLAPPING VS DISJOINT].

In our write-up, we will present a concise overview of the model, connections to neuroscience literature, and a comparison of Valiant's and Papadimitriou's algorithms. 

As a possible extension to our project, we can implement the algorithms proposed by Valiant and Papadimitriou and test their empirical performance on learning tasks described in the papers. We can also try to extend Papadimitriou's simulations and/or analysis to learning overlapping (in addition to disjoint) representations. For either extension, we can also include the results of our simulations, access to our implementations, and any insights gained from attempting to extend previous work to overlapping items.

\thispagestyle{empty}
%\doublespacing
\bstctlcite{bstctl:etal, bstctl:nodash, bstctl:simpurl}
\bibliographystyle{IEEEtranS}
\bibliography{COS511}

\end{document}

